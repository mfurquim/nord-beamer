% Quick start guide
\documentclass{beamer}

\RequirePackage{xcolor}
\usepackage{listings}
\usepackage{minted}
\usepackage{ulem}
\usepackage{bm}
\usepackage{textcomp}

\usepackage{tikz}
\usetikzlibrary{snakes}

\definecolor{NordBG}{HTML}{242933}

% set proper informations for pdf-file
\hypersetup{
	pdftitle=beamer-nord,
	pdfauthor=mfurquim,
	pdfsubject=presentation,
	hidelinks, % kill ugly link boxes
}

\usetheme{Nord}

\renewcommand{\familydefault}{FiraCode Mono}

\title{\LaTeX\ Beamer introduction}
\subtitle{Quick-start guide}
\author{mateus@mfurquim.dev}
\institute{MFurquim Dev}
\date{\today}


% \logo{
% \begin{figure}
% 	\def\svgwidth{\columnwidth}
% 	\scalebox{0.5}{\input{logo.pdf_tex}}
% \end{figure}
% }

% kill navigation symbols in presentation
\beamertemplatenavigationsymbolsempty

\setbeamertemplate{headline}
{%
	\vskip2pt\ \hskip2pt\tiny\textcolor{NordGreen}{\insertsection}\hfill\textcolor{NordBlue}{\insertsubsection}\hfill\ttfamily\insertpagenumber/\insertdocumentendpage\ \ \vskip2pt
}
\setbeamertemplate{footline}
{%
\begin{figure}
	\hfill
	\def\svgwidth{16px}
	\input{logo.pdf_tex}
	\hskip2pt
\end{figure}
}
% \setbeamertemplate{mini frames}[box]
% {\usebeamercolor{palette primary}}
% \setbeamertemplate{sidebar canvas}[vertical shading][top=primary.bg,middle=white,bottom=primary.bg]


\begin{document}
\fontsize{4pt}{1}\selectfont

% \setbeamercolor{background canvas}{bg=NordBG}

\usemintedstyle{nord}

\setminted[python]{texcomments=true,samepage=true,resetmargins=true,python3=true,tabsize=4,obeytabs=false,linenos=true,highlightcolor=NordBrightBlack,highlightlines={2, 5-9},fontsize=\tiny,autogobble,breaklines,breaksymbolleft=\textcolor{NordBrightBlack}{\tiny$\hookrightarrow$}}

\section{Exemplo de Seção 1}
\subsection{Exemplo de SubSeção 1}
\begin{frame}[fragile=singleslide,noframenumbering,squeeze]
	\inputminted{python}{example.py}
	\vspace*{-3px}
	\begin{tikzpicture}[overlay,remember picture,set style={{help lines}+=[very thin,NordBG]}]
		\draw[help lines] (0,0) grid (11,9);
		\draw [<->, rounded corners, thick, NordMagenta] (0,2) -- (0,0) -- (3,0);
		\draw[very thick, NordWhite] (0,0) to [out=90,in=195] (2,1.5);
		\draw [<->,thick, NordCyan] (0,0) to [out=90,in=180] (1,1)
		to [out=0,in=180] (2.5,0) to [out=0,in=-135] (4,1) ;
		\draw[NordRed, thick] (0,9) -- (11,0);
		\draw[NordBrightBlue, thick] (0,0) -- (11,9);
		\filldraw[NordBG] (5.5,4.5) circle (1pt);
		\node [above] at (5.5,4.5) {above};
		\node [below] at (5.5,4.5) {below};
		\node [left] at (5.5,4.5) {left};
		\node [right] at (5.5,4.5) {right};

		\filldraw[NordBG] (5.5,7.5) circle (1pt);
		\node [below right] at (5.5,7.5) {below right};
		\node [below left] at (5.5,7.5) {below left};
		\node [above right] at (5.5,7.5) {above right};
		\node [above left] at (5.5,7.5) {above left};

		\draw [thick]
		(5,9) -- (11,9);
		\draw (5,8.8) -- (5, 9.2);
		\draw (7,8.8) -- (7, 9.2);
		\draw (9,8.8) -- (9, 9.2);
		\draw (11,8.8) -- (11, 9.2);
		\begin{scope}[NordBlack]
			\node[align=left, below,fill=NordWhite,rounded corners] at (6,8.9)%
			{This happens\\in period 1\\and is aligned\\ left};
			\node[align=center, below,fill=NordWhite,rounded corners] at (8,8.9)%
			{This happens\\in period 2\\and is centered};
			\node[align=right, below,fill=NordWhite,rounded corners] at (10,8.9)%
			{This happens\\in period 3\\and is right aligned};
		\end{scope}

		\draw[x=10ex,y=10ex] (6.5,6.5) sin (8,7.5);

		\draw[snake=brace,thick] (0,3.73) -- (0.55,3.73);
		\node[align=left,above right,NordBlack,fill=NordWhite,rounded corners] at (0.32,3.93)%
		{Essa palavra chave indica a definição de uma classe};

		\draw[->,thick] (0.61,2.53) to [out=80,in=180] (0.87,2.6);
		\node[align=left,right,NordBlack,fill=NordWhite,rounded corners] at (0.87,2.66)%
		{Essa palavra chave indica a definição de uma função};

		% \filldraw[NordWhite] (0.61,2.53) circle (1pt);
		% \filldraw[NordWhite] (0.91,2.6) circle (1pt);
	\end{tikzpicture}


\end{frame}


\begin{frame}{Font size in Beamer (default)}

This is default font size

\tiny This is tiny font size

\scriptsize This is scriptsize font size

\footnotesize This is footnotesize font size

\small This is small font size

\normalsize This is normalsize font size

\large This is large font size

\Large This is Large font size

\LARGE This is LARGE font size

\huge This is huge font size

\Huge This is Huge font size

\end{frame}

\begin{frame}{Slanted and small caps text}
This is \textsc{small caps text} and this is
\textsl{slanted text}.\\~\\
You can combine them, to produce \textsl{\textsc{small
caps slanted text}} but also \textsc{\textbf{bold small caps}} or \textsl{\underline{underlined slanted text}}.\\~\\


\emph{This} is emphasized and \textit{\emph{this} is
also emphasized, although in a different way.}\\~\\


Let $\bm{u}$, $\bm{v}$ be vectors and $\bm{A}$ be a
matrix such that $\bm{Au}=\bm{v}$.
This is a bold integral:
\[
\bm{\int_{-\infty}^{\infty} e^{-x^2}\,dx=\sqrt{\pi} }
\]
\end{frame}

\begin{frame}{Text decorations provided by the \texttt{ulem} package}
\uline{Underlined that breaks at the end of lines if they are too too long because the author won’t stop writing.} \\~\\
\uuline{Double-Underlined text} \\~\\
\uwave{Wavy-Underlined text} \\~\\
\sout{Strikethrough text} \\~\\
\xout{Struck with Hatching text} \\~\\
\dashuline{Dashed Underline text} \\~\\
\dotuline{Dotted Underline text}
\end{frame}

\begin{frame}{Basic Blocks}
	\begin{block}{Standard Block}
		This is a standard block.
	\end{block}
	\begin{alertblock}{Alert Message}
		This block presents alert message.
	\end{alertblock}
	\begin{exampleblock}{An example of typesetting tool}
		Example: MS Word, \LaTeX{}
	\end{exampleblock}
\end{frame}
% Frame 2
\begin{frame}{Mathematical Environment Blocks}
	\begin{definition}
		This is a definition.
	\end{definition}

	\begin{theorem}
		This is a theorem.
	\end{theorem}

	\begin{lemma}
		This is a proof idea.
	\end{lemma}
\end{frame}
% Frame 3
\begin{frame}{Mathematical Environment Blocks-Continued}
	\begin{proof}
		This is a proof.
	\end{proof}

	\begin{corollary}
		This is a corollary
	\end{corollary}

	\begin{example}
		This is an example
	\end{example}
\end{frame}




\begin{frame}{Vertical line between columns}

	\begin{columns}[T]%Tcb
		% Column 1
		\begin{column}{0.49\textwidth}

			\begin{itemize}
				\item Input layer: 2 neurons.
				\item Hidden layer: 5 neurons.
				\item Output layer: 2 neurons.
			\end{itemize}

		\end{column}
		% Column 2 (vertical line)
		\begin{column}{.02\textwidth}
			\rule{.1mm}{0.7\textheight}
		\end{column}
		% Column 3
		\begin{column}{0.49\textwidth}
			\includegraphics[width=\textwidth]{logo_borders.png}
		\end{column}

	\end{columns}
\end{frame}

\begin{frame}[plain,squeeze]{Defined Colors}{This is a subtitle}
	\begin{description}[Snow Storm]
		\item[Polar Night]
			\textcolor{NordDarkBlack}{NordDarkBlack} \quad \textcolor{NordBlack}{NordBlack}\\
			\textcolor{NordMediumBlack}{NordMediumBlack} \quad \textcolor{NordBrightBlack}{NordBrightBlack}
		\item[Snow Storm]
			\textcolor{NordWhite}{NordWhite} \quad \textcolor{NordBrighterWhite}{NordBrightestWhite}\\
			\textcolor{NordBrightestWhite}{NordBrightestWhite}
		\item[Frost]
			\textcolor{NordCyan}{NordCyan} \quad \textcolor{NordBrightCyan}{NordBrightCyan}\\
			\textcolor{NordBlue}{NordBlue} \quad \textcolor{NordBrightBlue}{NordBrightBlue}
		\item[Aurora]
			\textcolor{NordRed}{NordRed} \quad \textcolor{NordOrange}{NordOrange} \\
			\textcolor{NordYellow}{NordYellow} \quad \textcolor{NordGreen}{NordGreen} \\
			\textcolor{NordMagenta}{NordMagenta}
	\end{description}
\end{frame}

\begin{frame}
\begin{tikzpicture}
\draw[NordRed, thick] (-1,2) -- (2,-4);
\draw[NordBrightBlue, thick] (-1,-1) -- (2,2);
\filldraw[black] (0,0) circle (2pt) node[anchor=west]{
		\includegraphics[width=0.5cm]{logo_borders.png}
};
\end{tikzpicture}
	% \begin{tikzpicture}[overlay,remember picture]
	% \node[right=0.2cm] at (current page.150) {%
	% 	\includegraphics[width=0.5cm]{logo_borders.png}
	% };
	% \end{tikzpicture}
\end{frame}

\begin{frame}[fragile]
\tiny\ttfamily
\begin{lstlisting}
#!/bin/bash
echo the $# parameter did not destroy pygments syntax highlighting
\end{lstlisting}
\end{frame}

\lstset{basicstyle=\footnotesize\ttfamily,breaklines=true}
\begin{frame}
	\lstinputlisting[language=Bash]{bash_script.sh}
\end{frame}


\begin{frame}

    This is your first presentation!
	\newline
    \rmfamily This is your first presentation!
	\newline
	\textup{Upright} \textit{Italic} \textsl{Slanted}
	\newline


    \sffamily This is your first presentation!
	\newline
	\textsf{Upright  \textit{Italic} \textsl{Slanted}}
	\newline

    \ttfamily This is your first presentation!
	\newline
	\texttt{Upright  \textit{Italic} \textsl{Slanted}}
	\newline

	\textsc{Small Capitals}

\end{frame}

% \begin{overlayarea}<overlay spec>{area width}{area height}
\begin{overlayarea}{1080px}{1080px}
  content
\end{overlayarea}

\end{document}
